\documentclass[12pt,a4paper,oneside,openany]{report}

\usepackage{verbatim}

\setlength\paperwidth{210mm}
\setlength\paperheight{297mm}
\setlength\parindent{0mm}

\setlength{\voffset}{-25.4mm}
\setlength{\hoffset}{-25.4mm}

\setlength{\topmargin}{20mm}
\setlength{\oddsidemargin}{20mm}
\setlength{\evensidemargin}{20mm}
\setlength{\textwidth}{160mm}
\setlength{\textheight}{247mm}

\setlength\footskip{10mm}
\setlength\headheight{0cm}
\setlength\headsep{0cm}
\pagestyle{plain}
\pagenumbering{arabic}
\parskip 5mm

\newcommand{\fcode}[1]{\par file = \textbf{#1} \verbatiminput{example/#1}}
\newcommand{\code}[1]{\texttt{#1}}
\newcommand{\SC}{\textbf{SC} }

\begin{document}

\section*{Parallel N-body assignment}

In laboratory exercise 2, you wrote a simple N-body code to simulate $n$ particles in a box interacting via the coulomb force
\begin{equation}
F=\frac{q Q}{4\pi\epsilon_{0} \mathbf{r}^{2}}
\end{equation}

Expand this simple N-body code to run in parallel, using MPI.  Provide a brief discussion on how you parallelised the algorithm and why, generally, you wouldn't use this algorithm in practice.

Please email me your code and the brief discussion, outlined above, to \code{stuart.midgley@anu.edu.au}.  Include all files and instructions on how to run your code.

\textbf{Due 29 August}


\end{document}